%%%%%%%%%%%%%%%%%%%%%%%%%%%%%%%%%%%%%%%%%%%%%%%%%%%%%%%%%%%%%%%%%
% Шаблон презентации НТУ "Сириус"
% Автор: на основе шаблона от НТУ "Сириус"
%%%%%%%%%%%%%%%%%%%%%%%%%%%%%%%%%%%%%%%%%%%%%%%%%%%%%%%%%%%%%%%%%

%----------------------------------------------------------
% Настройка документа и базовых пакетов
%----------------------------------------------------------
\documentclass[10pt]{beamer}  % Класс для создания презентаций, размер шрифта 10pt

% Настройка для XeLaTeX и поддержки русского языка
\usepackage{fontspec}
\usepackage{polyglossia}
\setdefaultlanguage{russian}
\setotherlanguage{english}

% Настройка шрифтов
\defaultfontfeatures{Ligatures=TeX}
\setmainfont{Times New Roman}
\setsansfont{Arial}
\setmonofont{Courier New}

% Дополнительные пакеты
\usepackage{animate}          % Поддержка анимации
\usepackage{xcolor}           % Работа с цветом
\usepackage[absolute,overlay]{textpos}  % Позиционирование текста
\usepackage{amsmath}          % Математические формулы
\usepackage{fontawesome}      % Иконки
\usepackage{tikz}             % Рисование векторной графики

% Настройка библиографии
\setbeamertemplate{bibliography item}{\insertbiblabel}

%----------------------------------------------------------
% Настройка темы и цветов презентации
%----------------------------------------------------------
\usetheme{default}            % Базовая тема Beamer

% Определение фирменного цвета Сириуса
\definecolor{Sirius}{rgb}{0.004, 0.616, 0.631} 

% Установка цвета структурных элементов в фирменный цвет
\usecolortheme[named=Sirius]{structure}

% Добавление номера слайда в нижний колонтитул
\setbeamertemplate{footline}[frame number]

%----------------------------------------------------------
% Настройка метаданных презентации
%----------------------------------------------------------
\hypersetup{unicode=true}     % Для корректной работы с Unicode в метаданных

\title[Краткий заголовок]{\bf 
Название презентации}  % Заголовок презентации

\date[]{}  % Дата (пустая по умолчанию)

\author[Автор] {\bf 
ФИО автора \\ 
Должность/звание \\ 
Дополнительная информация}

\institute[] {
НТУ <<Сириус>> \\ 
Название направления \\ 
Название лаборатории
}

%----------------------------------------------------------
% Настройка логотипа
%----------------------------------------------------------
\logo{
\includegraphics[width=1.5cm]{images/Sirius-logo.png}
}

%----------------------------------------------------------
% Определение математических команд
%----------------------------------------------------------
\newcommand{\bF}{\mathbb{F}}
\newcommand{\bC}{\mathbb{C}}
\newcommand{\bP}{\mathbb{P}}
\newcommand{\bN}{\mathbb{N}}
\newcommand{\bs}{\mathbb{S}}

%----------------------------------------------------------
% Начало документа
%----------------------------------------------------------
\begin{document}

%----------------------------------------------------------
% Титульный слайд
%----------------------------------------------------------
\maketitle

%----------------------------------------------------------
% Пример слайда с двумя изображениями
%----------------------------------------------------------
\begin{frame}{Название слайда}

\begin{figure}
    \begin{minipage}{0.45\textwidth}
        \centering
        \includegraphics[width=\linewidth]{images/Sirius-logo.png}
        \caption{Подпись к первому изображению}
    \end{minipage}
    \hfill
    \begin{minipage}{0.45\textwidth}
        \centering
        \includegraphics[width=\linewidth]{images/Sirius-logo.png}
        \caption{Подпись ко второму изображению}
    \end{minipage}
\end{figure}

\end{frame}

%----------------------------------------------------------
% Пример слайда с формулами
%----------------------------------------------------------
\begin{frame}{Пример математических формул}
    \begin{equation}
        \label{eq:example}
        \begin{cases}
             \text{div}_x \bP + \boldsymbol{f} = \rho \frac{\partial^2 \boldsymbol{u}}{\partial t^2}  & \text{в области } \mathcal{B}, \\
            \bP \cdot \boldsymbol{n} = \boldsymbol{g} & \text{на границе } \partial \mathcal{B}_g, \\
            \boldsymbol{u} = \boldsymbol{u}_0 & \text{на границе } \partial \mathcal{B}_u.
        \end{cases}
    \end{equation}
    
    \begin{itemize}
        \item $\displaystyle \bP = \frac{\boldsymbol{f}_{int}}{\Delta A}$ -- тензор напряжений
        \item Формула с ссылкой на уравнение \eqref{eq:example}
    \end{itemize}
\end{frame}

%----------------------------------------------------------
% Пример слайда с маркированным списком
%----------------------------------------------------------
\begin{frame}{Пример маркированного списка}
    \begin{itemize}
        \item Первый элемент списка
        \item Второй элемент списка
            \begin{itemize}
                \item Вложенный элемент
                \item Еще один вложенный элемент
            \end{itemize}
        \item Третий элемент списка
    \end{itemize}
\end{frame}

%----------------------------------------------------------
% Пример слайда с нумерованным списком
%----------------------------------------------------------
\begin{frame}{Пример нумерованного списка}
    \begin{enumerate}
        \item Первый шаг
        \item Второй шаг
        \item Третий шаг
    \end{enumerate}
\end{frame}

%----------------------------------------------------------
% Пример слайда с цитатой/ссылкой
%----------------------------------------------------------
\begin{frame}{Пример с цитированием}
    Основной текст слайда с важной информацией.
    
    \vspace{1em}
    
    \tiny Источник: Фамилия И.О. и др. Название статьи // Журнал. – Год. – Т. X. – С. XX-XX.
\end{frame}

%----------------------------------------------------------
% Пример слайда с большим изображением на весь экран
%----------------------------------------------------------
\begin{frame}[plain]
    \frametitle{Изображение на весь экран}
    \begin{tikzpicture}[remember picture, overlay]
        \node[at=(current page.center)] {
            \includegraphics[width=\paperwidth,height=\paperheight,keepaspectratio]{images/coffee.png}
        };
    \end{tikzpicture}
\end{frame}

%----------------------------------------------------------
% Пример слайда с двумя колонками
%----------------------------------------------------------
\begin{frame}{Две колонки}
    \begin{columns}
        \begin{column}{0.48\textwidth}
            \textbf{Левая колонка:}
            \begin{itemize}
                \item Пункт 1
                \item Пункт 2
                \item Пункт 3
            \end{itemize}
        \end{column}
        \begin{column}{0.48\textwidth}
            \textbf{Правая колонка:}
            \begin{figure}
                \centering
                \includegraphics[width=\linewidth]{images/Sirius-logo.png}
                \caption{Подпись к изображению}
            \end{figure}
        \end{column}
    \end{columns}
\end{frame}

%----------------------------------------------------------
% Слайд с благодарностью
%----------------------------------------------------------
\begin{frame}
    \centering
    \LARGE Спасибо за внимание!
    
    \vspace{1em}
    
    \large Вопросы?
\end{frame}

\end{document} 